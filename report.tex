%%%% Proceedings format for most of ACM conferences (with the exceptions listed below) and all ICPS volumes.
\documentclass[sigconf]{acmart}
%%%% As of March 2017, [siggraph] is no longer used. Please use sigconf (above) for SIGGRAPH conferences.

%%%% Proceedings format for SIGPLAN conferences 
% \documentclass[sigplan, anonymous, review]{acmart}

%%%% Proceedings format for SIGCHI conferences
% \documentclass[sigchi, review]{acmart}

%%%% To use the SIGCHI extended abstract template, please visit
% https://www.overleaf.com/read/zzzfqvkmrfzn


\usepackage{booktabs} % For formal tables

% Macros
\makeatletter % changes the catcode of '@' to 11
\newcommand\etal{et al\@ifnextchar.{}{.\@}}
\newcommand\etc{etc\@ifnextchar.{}{.\@}}
\newcommand\ie{i.e.\@}
\newcommand\eg{e.g.\@}
\makeatother % restores the catcode of '@' to 12

% Copyright
%\setcopyright{none}
%\setcopyright{acmcopyright}
%\setcopyright{acmlicensed}
\setcopyright{rightsretained}
%\setcopyright{usgov}
%\setcopyright{usgovmixed}
%\setcopyright{cagov}
%\setcopyright{cagovmixed}


% DOI
% \acmDOI{10.475/123_4}

% ISBN
% \acmISBN{123-4567-24-567/08/06}

%Conference
% \acmConference[WOODSTOCK'97]{ACM Woodstock conference}{July 1997}{El
%   Paso, Texas USA}
% \acmYear{1997}
% \copyrightyear{2016}


% \acmArticle{4}
% \acmPrice{15.00}

% These commands are optional
% \acmBooktitle{Transactions of the ACM Woodstock conference}
% \editor{Jennifer B. Sartor}
% \editor{Theo D'Hondt}
% \editor{Wolfgang De Meuter}


\begin{document}
\title{Augmenting the PIT Mutation Testing Tool by Implementing More Mutation operators and Code-Fixing Rules}


\author{Khoa Nguyen}
\affiliation{%
  \institution{The University of Texas at Dallas}
}
\email{kxn161730@utdallas.edu}

\author{Leejia James}
\affiliation{%
  \institution{The University of Texas at Dallas}
}
\email{lxj171130@utdallas.edu}

\author{Joseph LaFreniere}
\affiliation{%
  \institution{The University of Texas at Dallas}
}
\email{lafrenierejm@utdallas.edu}

\begin{abstract}
This paper provides a sample of a \LaTeX\ document which conforms,
somewhat loosely, to the formatting guidelines for
ACM SIG Proceedings.\footnote{This is an abstract footnote}
\end{abstract}

%
% The code below should be generated by the tool at
% http://dl.acm.org/ccs.cfm
%
\begin{CCSXML}
  <ccs2012>
  <concept>
  <concept_id>10011007.10011074.10011099.10011102.10011103</concept_id>
  <concept_desc>Software and its engineering~Software testing and debugging</concept_desc>
  <concept_significance>500</concept_significance>
  </concept>
  </ccs2012>
\end{CCSXML}

\ccsdesc[500]{Software and its engineering~Software testing and debugging}

\keywords{ACM proceedings, \LaTeX, text tagging}


\maketitle

\section{Introduction}

PIT is a state-of-the-art mutation testing system for the Java programming language and Java’s underlying JVM.
PIT provides built-in mutators, of which most are activated by default.
However, there are some traditional mutation operators that PIT currently lacks.
This project will be implemented and delivered in two phases.
The first phase will be delivered 2018-03-21.
This phase aims to augment the PIT mutation testing tool by implementing three new mutation operators:
\begin{itemize}
\item
  Arithmetic Operator Deletion (AOD):
  Replaces an arithmetic expression by each one of the operands.
  For example, \texttt{a + b} is mutated to both \texttt{a} and \texttt{b}.
\item
  Relational Operator Replacement (ROR):
  Replaces the relational operators with each of the other ones.
  For example, \texttt{<} is mutated to each of \texttt{>=}, \texttt{<=}, and \texttt{!=}.
\item
  Arithmetic Operator Replacement (AOR):
  Replaces an arithmetic expression by each of the other ones.
  For example, \texttt{a + b} is mutated to each of \texttt{a - b}, \texttt{a * b}, \texttt{a / b}, and \texttt{a \% b}.
\end{itemize}

The second phase of the project will be delivered 2018-04-28.  This phase will add four additional mutators to PIT:
\begin{itemize}
\item
  M1:
  For each object field dereference, add a conditional checker to perform the check only when the object exists (\ie{} is not \texttt{null}).
\item
  M2:
  Replace a method with another overloading method or overloading constructor.
  If the method or constructor has fewer arguments, use a subset of the arguments.
  If there are more arguments in the overloaded method, add \texttt{null} or a default value as the additional argument values.
  One possible improvement is to shuffle the extra argument among the set.
\item
  M3:
  Replace a method with an existing method of a different name but the same return value and arguments.
\item
  M4:
  Replace local variables within each expression.
\end{itemize}

The second phase will also involve gathering experimental data on the the effectiveness of the added mutators in discovering and removing defects from open source codebases.


\bibliographystyle{ACM-Reference-Format}
\bibliography{bibliography.bib}

\end{document}
