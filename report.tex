%%%% Proceedings format for most of ACM conferences (with the exceptions listed below) and all ICPS volumes.
\documentclass[sigconf]{acmart}
%%%% As of March 2017, [siggraph] is no longer used. Please use sigconf (above) for SIGGRAPH conferences.

%%%% Proceedings format for SIGPLAN conferences 
% \documentclass[sigplan, anonymous, review]{acmart}

%%%% Proceedings format for SIGCHI conferences
% \documentclass[sigchi, review]{acmart}

%%%% To use the SIGCHI extended abstract template, please visit
% https://www.overleaf.com/read/zzzfqvkmrfzn


\usepackage{booktabs} % For formal tables

% Source code
\usepackage{listingsutf8}
%% Java
\lstnewenvironment{Java}{%
  \lstset{%
    columns=fixed,%
    language=java,%
  }
}{}
\newcommand{\java}[1]{%
  \lstinline[%
    columns=fixed,%
    language=java
  ]{#1}
}
%% JVM
\lstnewenvironment{JVM}{%
  \lstset{%
    columns=fixed,%
    language={},%
  }
}{}
\newcommand{\jvm}[1]{%
  \lstinline[
    columns=fixed,%
    language={},%
  ]{#1}
}

% Macros
\makeatletter % changes the catcode of '@' to 11
\newcommand\etal{et al\@ifnextchar.{}{.\@}}
\newcommand\etc{etc\@ifnextchar.{}{.\@}}
\newcommand\ie{i.e.\@}
\newcommand\eg{e.g.\@}
\makeatother % restores the catcode of '@' to 12

% Copyright
%\setcopyright{none}
%\setcopyright{acmcopyright}
%\setcopyright{acmlicensed}
\setcopyright{rightsretained}
%\setcopyright{usgov}
%\setcopyright{usgovmixed}
%\setcopyright{cagov}
%\setcopyright{cagovmixed}


% DOI
% \acmDOI{10.475/123_4}

% ISBN
% \acmISBN{123-4567-24-567/08/06}

%Conference
% \acmConference[WOODSTOCK'97]{ACM Woodstock conference}{July 1997}{El
%   Paso, Texas USA}
% \acmYear{1997}
% \copyrightyear{2016}


% \acmArticle{4}
% \acmPrice{15.00}

% These commands are optional
% \acmBooktitle{Transactions of the ACM Woodstock conference}
% \editor{Jennifer B. Sartor}
% \editor{Theo D'Hondt}
% \editor{Wolfgang De Meuter}


\begin{document}
\title{Augmenting the PIT Mutation Testing Tool}
\subtitle{Implementing New Mutation Operators and Code-Fixing Rules}


\author{Khoa Nguyen}
\affiliation{%
  \institution{The University of Texas at Dallas}
}
\email{kxn161730@utdallas.edu}

\author{Leejia James}
\affiliation{%
  \institution{The University of Texas at Dallas}
}
\email{lxj171130@utdallas.edu}

\author{Joseph LaFreniere}
\affiliation{%
  \institution{The University of Texas at Dallas}
}
\email{lafrenierejm@utdallas.edu}

\begin{abstract}
PIT is a mutation testing system, providing best test coverage for Java and the JVM.
PIT provides built-in mutators, of which most are activated by default.
However, it does not include some traditional mutation operators.
This project aims to augment the PIT mutation testing tool to implement some such mutation operators and to evaluate the augmented PIT tool on real-world projects from GitHub.com.
Augmenting the PIT tool with more code-fixing rules helps to automatically fix real-world program bugs from defects.
This is the other goal of this project.
\end{abstract}

%
% The code below should be generated by the tool at
% http://dl.acm.org/ccs.cfm
%
\begin{CCSXML}
  <ccs2012>
  <concept>
  <concept_id>10011007.10011074.10011099.10011102.10011103</concept_id>
  <concept_desc>Software and its engineering~Software testing and debugging</concept_desc>
  <concept_significance>500</concept_significance>
  </concept>
  </ccs2012>
\end{CCSXML}

\ccsdesc[500]{Software and its engineering~Software testing and debugging}

\keywords{PIT, mutator, mutation testing, JVM, bytecode}


\maketitle

\section{Introduction}

PIT is a state-of-the-art mutation testing system for the Java programming language and Java’s underlying JVM.
PIT provides built-in mutators, of which most are activated by default.
However, there are some traditional mutation operators that PIT currently lacks.
This project will be implemented and delivered in two phases.
The first phase will be delivered 2018-03-21.
This phase aims to augment the PIT mutation testing tool by implementing three new mutation operators:
\begin{itemize}
\item
  Arithmetic Operator Deletion (AOD):
  Replaces an arithmetic expression by each one of the operands.
  For example, \texttt{a + b} is mutated to both \texttt{a} and \texttt{b}.
\item
  Relational Operator Replacement (ROR):
  Replaces the relational operators with each of the other ones.
  For example, \texttt{<} is mutated to each of \texttt{>=}, \texttt{<=}, and \texttt{!=}.
\item
  Arithmetic Operator Replacement (AOR):
  Replaces an arithmetic expression by each of the other ones.
  For example, \texttt{a + b} is mutated to each of \texttt{a - b}, \texttt{a * b}, \texttt{a / b}, and \texttt{a \% b}.
\end{itemize}

The second phase of the project will be delivered 2018-04-28.  This phase will add four additional mutators to PIT:
\begin{itemize}
\item
  M1:
  For each object field dereference, add a conditional checker to perform the check only when the object exists (\ie{} is not \texttt{null}).
\item
  M2:
  Replace a method with another overloading method or overloading constructor.
  If the method or constructor has fewer arguments, use a subset of the arguments.
  If there are more arguments in the overloaded method, add \texttt{null} or a default value as the additional argument values.
  One possible improvement is to shuffle the extra argument among the set.
\item
  M3:
  Replace a method with an existing method of a different name but the same return value and arguments.
\item
  M4:
  Replace local variables within each expression.
\end{itemize}

The second phase will also involve gathering experimental data on the the effectiveness of the added mutators in discovering and removing defects from open source codebases.

\section{First Phase}

\subsection{Arithmetic Operator Deletion}
\subsection{Relational Operator Replacement}

\begin{table}
  \centering
  \begin{tabular}{c c}
    \toprule
    Original Operator & Mutated Operators                                                        \\
    \midrule
    \java{>}          & \phantom{\java{>},} \java{>=}, \java{==}, \java{<=}, \java{<}, \java{!=} \\
    \java{>=}         & \java{>}, \phantom{\java{>=},} \java{==}, \java{<=}, \java{<}, \java{!=} \\
    \java{==}         & \java{>}, \java{>=}, \phantom{\java{==},} \java{<=}, \java{<}, \java{!=} \\
    \java{<=}         & \java{>}, \java{>=}, \java{==}, \phantom{\java{<=},} \java{<}, \java{!=} \\
    \java{<}          & \java{>}, \java{>=}, \java{==}, \java{<=}, \phantom{\java{<},} \java{!=} \\
    \java{!=}         & \java{>}, \java{>=}, \java{==}, \java{<=}, \java{<}\phantom{, \java{!=}} \\
    \bottomrule
  \end{tabular}
  \caption[ROR mutations]{Relational operator replacement mutations}
  \label{tab:ror:src_ops}
\end{table}

The relational operator replacement (ROR) mutator performs bytecode mutations equivalent to the source code mutations listed in Table~\ref{tab:ror:src_ops}.
Each of the source-level ordered inequality operators (\java{>}, \java{>=}, \java{==}, \java{<=}, and \java{<}) has two corresponding bytecode opcodes, one that has two operands and one that compares a single operand to the zero value.
For example, the \java{!=} operator in source code can be compiled to the opcode \path{IF_ICMPNE} or \path{IFNE} that check for inequality between two operands and check for inequality between a single operand and zero, respectively.
The ROR mutator must use all of the bytecode-level ordered inequality operators in its implementation of \java{MethodVisitor}.

Other implementations of ROR mutators manually list every replacement opcode for every source-level relational operator~\cite{ProdigyXable}.
This results in a large amount of code duplication, much of which is unavoidable due to the \java{Map} returned being \java{static}.
However, we were able to reduce the number of changes required between each implementation by storing the bytecode relational operators in enums and iterating over those operators, adding to the \java{Map} only when the loop's current opcode is not the opcode being replaced.
Listing~\ref{lst:ror:enum} shows an excerpt of the enumeration of the zero-comparing opcodes.
Listing~\ref{lst:ror:MethodVisitor} shows one of the enum iterations from the ROR mutator's \java{MethodVisitor}.

\begin{lstlisting}[%
  language=Java,
  frame=tb,
  caption={%
    enum excerpt from the ROR mutator.
    The \java{Opcodes} enum is imported from the package \java{org.objectweb.asm.Opcodes}.
  },
  captionpos=b,
  label={lst:ror:enum}]
enum OpcodeCompareToZero {
  IFEQ(Opcodes.IFEQ) {
    public String toString() {
      return "==";
    }
  },
  // ...
  IFNE(Opcodes.IFNE) {
    public String toString() {
      return "!=";
    }
  };

  private final int opcode;

  OpcodeCompareToZero(int opcode) {
    this.opcode = opcode;
  }

  public int getOpcode() {
    return this.opcode;
  }
}
\end{lstlisting}

\begin{lstlisting}[%
  language=Java,
  frame=tb,
  caption={%
    Excerpt of the implementation of \java{MethodVisitor} for the ROR mutator.
    The value of \java{REPLACEMENT_OP} has been previously declared to be an opcode.
  },
  captionpos=b,
  label={lst:ror:MethodVisitor}]
// The operands will seem to be in the wrong
// order when used in else conditions.
// To the bytecode parser, though, this is
// not the case.
for (OpcodeCompareToZero original
    : OpcodeCompareToZero.values()) {
  if (REPLACEMENT_OP != original) {
    MUTATIONS.put(
      original.getOpcode(),
      new Substitution(
        REPLACEMENT_OP.getOpcode(),
        "Relational operator replacement: "
        + "Mutated " + original
        + " to " + REPLACEMENT_OP));
  }
}
\end{lstlisting}

\subsection{Arithmetic Operator Replacement}
The mutation operator Arithmetic Operator Replacement (AOR) replaces basic arithmetic operators with other arithmetic operators.
Pitest mutation testing tool implements only a subset of AOR, which is available within Math Mutator.
The replacements available in Math Mutator are given in Table~\ref{tab:math:src_ops}.

\begin{table}
  \centering
  \begin{tabular}{c c}
    \toprule
    Original Operator & Mutated Operator \\
    \midrule
    \java{+}          & \java{-}         \\
    \java{-}          & \java{+}         \\
    \java{*}          & \java{/}         \\
    \java{/}          & \java{*}         \\
    \java{\%}         & \java{*}         \\
    \bottomrule
  \end{tabular}
  \caption[Math mutations]{Math mutations}
  \label{tab:math:src_ops}
\end{table}

In Phase One of this project, all missing AOR mutators are implemented.
The entire replacements available in augmented PIT are summarized in Table~\ref{tab:aor:src_ops}.

\begin{table}
  \centering
  \begin{tabular}{c c}
    \toprule
    Original Operator & Mutated Operators                                           \\
    \midrule
    \java{+}          & \phantom{\java{+},} \java{-}, \java{*}, \java{/}, \java{\%} \\
    \java{-}          & \java{+}, \phantom{\java{-},} \java{*}, \java{/}, \java{\%} \\
    \java{*}          & \java{+}, \java{-}, \phantom{\java{*},} \java{/}, \java{\%} \\
    \java{/}          & \java{+}, \java{-}, \java{*}, \phantom{\java{/},} \java{\%} \\
    \java{\%}         & \java{+}, \java{-}, \java{*}, \java{/}\phantom{, \java{\%}} \\
    \bottomrule
  \end{tabular}
  \caption[AOR mutations]{Arithmetic operator replacement mutations}
  \label{tab:aor:src_ops}
\end{table}

We have implemented integer, double, float and long versions of above arithmetic mutators.



The augmented Pitest is run on 5 real world projects from GitHub and is evaluated for its functionality.
We could see increased mutation coverage and line coverage in Project Summary report available in target/pit-reports directory of the project.
We also analyzed the additional mutations generated by AOD, AOR, and ROR in different java source files and the status of the mutations, that is, Survived, Killed or No Coverage.

\section{Notable Problems}
\subsection{AOD With More Than Two Operands}
In an expression such as \java{a + b + c}, it is possible to apply AOD mutation to retain \java{a + b} and \java{c}; \java{b + c} and \java{a}; and \java{a + c} and \java{b}.
However, the current AOD implementation recursively returns \java{a}, \java{b}, \java{c} operands separately.
Adding handling of consecutive arithmetic operations may be considered for future improvements, but at the moment we do not see the significance of these mutations;
they will add more mutations to evaluate the tests but might not give better evaluation.

\subsection{Grouping The Implemented Mutations}
To register the mutations, we have to add them to Mutator.java. For AOD and AOR, this is not a problem. However for ROR, we need to test them invididually and split the enum of ROR into several parts. It is possible to group all the ENUM into one class so we do not have to add many mutations to the pom.xml files of test projects.

\subsection{Grouping the Mutators}
To register the mutations, we have to add them to \path{Mutator.java}.
This is trivial for AOD and AOR.
However for ROR, we need to test each mutator individually and split the enum of ROR into several parts.
It is possible to group all the enums into one class so we do not have to add many mutations to the \path{pom.xml} files of test projects.

%%% Local Variables:
%%% mode: latex
%%% TeX-master: "../report"
%%% End:

\section{Second Phase}

\subsection{Increments Increments Mutator}
\textbf{Existing work}\newline
\textbf{Increments Mutator}\newline
The increments mutator will mutate increments, decrements and assignment increments and decrements of local variables (stack variables). It will replace increments with decrements and vice versa. This mutator is Active by default. Also the increments mutator will be applied to increments of local variables only.\newline


For example, \newline
\begin{lstlisting}[%
  language=Java,
  label={lst:iim:inctodec}]
public int method(int i) {
  i++;
  return i;
}
\end{lstlisting}

will be mutated to \newline

\begin{lstlisting}[%
  language=Java,
  label={lst:iim:dectoinc2}]
public int method(int i) {
  i--;
  return i;
}
\end{lstlisting}
\textbf{Remove Increments Mutator}\newline
The remove increments mutator removes local variable increments. This mutator is an optional mutator and will be applied to increments of local variables only.\newline


For example, \newline
\begin{lstlisting}[%
  language=Java,
  label={lst:iim:reminc1}]
public int method(int i) {
  i++;
  return i;
}
\end{lstlisting}

will be mutated to \newline

\begin{lstlisting}[%
  language=Java,
  label={lst:iim:reminc2}]
public int method(int i) {
  //i;
  return i;
}
\end{lstlisting}

\textbf{Augmentation}\newline
\textbf{Increments Increments Mutator}\newline
The increments increments mutator increments local variable increments. This mutator is an optional mutator and will be applied to increments of local variables only.\newline


For example, \newline
\begin{lstlisting}[%
  language=Java,
  label={lst:iim:incinc1}]
public int method(int i) {
  i++;
  return i;
}
\end{lstlisting}

will be mutated to \newline

\begin{lstlisting}[%
  language=Java,
  label={lst:iim:incinc2}]
public int method(int i) {
  i+=2;
  return i;
}
\end{lstlisting}



%%% Local Variables:
%%% mode: latex
%%% TeX-master: "../report"
%%% End:


\bibliographystyle{ACM-Reference-Format}
\bibliography{bibliography.bib}

\end{document}
