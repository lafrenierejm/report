\subsection{Arithmetic Operator Deletion}
The structure of a simple operation with two operands looks like this:
An example is: 
\begin{itemize}
\item IADD
\item ILOAD 2
\item ILOAD 1
\end{itemize}

Note that if we use the bytecode analysis tool in eclipse, the stack will be inverted. The first operation is removing the operator by using POP or POP2. 

To pick the first operator, we simply use POP until we get to the first operator. For double-word operands and double-word operators, we use POP2.To keep the second operand while removing the first operator, we have to use SWAP and then POP. However, this only works for single-word operands (int and float). For double-word operands, we do not have SWAP2 in ASM framework. Therefore, we have to combine DUP2\_X2 and POP2. Here's how the operation works:

\begin{itemize}
\item LLOAD 1 - word 1
\item LLOAD 1 - word 2
\item LLOAD 2 - word 1
\item LLOAD 2 - word 2
\end{itemize}

After using DUP2\_x2, we have:
\begin{itemize}
\item LLOAD 1 - word 1
\item LLOAD 1 - word 2
\item LLOAD 2 - word 1
\item LLOAD 2 - word 2
\item LLOAD 1 - word 1
\item LLOAD 1 - word 2
\end{itemize}

Then, using POP2 twice, we are left with 
\begin{itemize}
\item LLOAD 1 - word 1
\item LLOAD 1 - word 2
\end{itemize}