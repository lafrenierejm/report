\subsection{M1: Check Null Mutator}
M1, the Check Null Mutator, inserts null checks on objects prior to field access instructions.
This has practical uses in generating fixes for unexpected \java{NullPointerException}s.

The two opcodes that indicate a field access are \jvm{PUTFIELD} and \jvm{GETFIELD} which write to and read from an object's field, respectively.
M1 executes those opcodes only if the object whose field they are accessing is not null.
The general algorithm for our implementation of M1 is:
\begin{enumerate}
\item
  Duplicate the objectref on top of the stack
\item
  Load a \java{null} reference onto the stack
\item
  Compare the references' equality
\end{enumerate}
\begin{minipage}[t]{0.45\linewidth}
  If the references are the same (\ie{} \java{null}):
  \begin{enumerate}[%
    leftmargin=*,%
    label={(\arabic*a)}
    ]
    \setcounter{enumi}{3}
  \item
    Skip the original opcode
  \item
    Pop the unneeded object reference from the stack
  \item\label{itm:m1:load default}
    If the original opcode was \jvm{GETFIELD}, load a default value onto the stack.
  \item
    Return to normal execution
  \end{enumerate}
\end{minipage}
\hfill\vline\hfill
\begin{minipage}[t]{0.5\linewidth}
  If the references are \textit{not} the same (\ie{} are not \java{null}):
  \begin{enumerate}[%
    leftmargin=*,%
    label={(\arabic*b)}
    ]
    \setcounter{enumi}{3}
  \item\label{itm:m1:execute original}
    Execute the original field access opcode
  \item\label{itm:m1:return}
    Return to normal execution
  \end{enumerate}
\end{minipage}

\vspace*{2ex}

M1's logic is implemented in \java{visitFieldInsn()}.
Two labels are used in the implementation.
The first is placed immediately prior to the opcodes implementing the (a) branch above.
This label is jumped to conditionally with \jvm{IF_ACMPEQ}, whose arguments are a duplicate of the original object reference and a \java{null} reference.
If that jump is not performed then the execution is allowed to fall through to \ref{itm:m1:execute original}.
A second label is placed immediately \textit{after} \ref{itm:m1:load default}.
This label is jumped to unconditionally with a \jvm{GOTO} in \ref{itm:m1:return}.

%%% Local Variables:
%%% mode: latex
%%% TeX-master: "../../report"
%%% End:
