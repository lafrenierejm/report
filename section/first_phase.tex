\section{First Phase}

\subsection{Arithmetic Operator Deletion}
The structure of a simple operation with two operands looks like this:
An example is: 
\begin{itemize}
\item IADD
\item ILOAD 2
\item ILOAD 1
\end{itemize}

Note that if we use the bytecode analysis tool in eclipse, the stack will be inverted. The first operation is removing the operator by using POP or POP2. 

To pick the first operator, we simply use POP until we get to the first operator. For double-word operands and double-word operators, we use POP2.To keep the second operand while removing the first operator, we have to use SWAP and then POP. However, this only works for single-word operands (int and float). For double-word operands, we do not have SWAP2 in ASM framework. Therefore, we have to combine DUP2\_X2 and POP2. Here's how the operation works:

\begin{itemize}
\item LLOAD 1 - word 1
\item LLOAD 1 - word 2
\item LLOAD 2 - word 1
\item LLOAD 2 - word 2
\end{itemize}

After using DUP2\_x2, we have:
\begin{itemize}
\item LLOAD 1 - word 1
\item LLOAD 1 - word 2
\item LLOAD 2 - word 1
\item LLOAD 2 - word 2
\item LLOAD 1 - word 1
\item LLOAD 1 - word 2
\end{itemize}

Then, using POP2 twice, we are left with 
\begin{itemize}
\item LLOAD 1 - word 1
\item LLOAD 1 - word 2
\end{itemize}
\subsection{Relational Operator Replacement}

\begin{table}
  \centering
  \begin{tabular}{c c}
    \toprule
    Original Operator & Mutated Operators                                                        \\
    \midrule
    \java{>}          & \phantom{\java{>},} \java{>=}, \java{==}, \java{<=}, \java{<}, \java{!=} \\
    \java{>=}         & \java{>}, \phantom{\java{>=},} \java{==}, \java{<=}, \java{<}, \java{!=} \\
    \java{==}         & \java{>}, \java{>=}, \phantom{\java{==},} \java{<=}, \java{<}, \java{!=} \\
    \java{<=}         & \java{>}, \java{>=}, \java{==}, \phantom{\java{<=},} \java{<}, \java{!=} \\
    \java{<}          & \java{>}, \java{>=}, \java{==}, \java{<=}, \phantom{\java{<},} \java{!=} \\
    \java{!=}         & \java{>}, \java{>=}, \java{==}, \java{<=}, \java{<}\phantom{, \java{!=}} \\
    \bottomrule
  \end{tabular}
  \caption[ROR mutations]{Relational operator replacement mutations}
  \label{tab:ror:src_ops}
\end{table}

\begin{table}
  \centering
  \begin{tabular}{l l l}
    \toprule
    Java Operator & \multicolumn{2}{c}{JVM Opcode}       \\
    \cmidrule(l){2-3}
                  & Compare \java{0} & Compare Stack     \\
    \midrule
    \java{>}      & IFGT     & \jvm{IF_ICMPGT} \\
    \java{>=}     & IFGE     & \jvm{IF_ICMPGE} \\
    \java{==}     & IFEQ     & \jvm{IF_ICMPEQ} \\
    \java{<=}     & IFLE     & \jvm{IF_ICMPLE} \\
    \java{<}      & IFLT     & \jvm{IF_ICMPLT} \\
    \java{!=}     & IFNE     & \jvm{IF_ICMPNE} \\
    \bottomrule
  \end{tabular}
  \caption[Relational Opcodes]{Mapping Relational Operators to Opcodes}
  \label{tab:ror:opcode}
\end{table}

The relational operator replacement (ROR) mutator performs bytecode mutations equivalent to the source code mutations listed in Table~\ref{tab:ror:src_ops}.
Each of the source-level ordered inequality operators (\java{>}, \java{>=}, \java{==}, \java{<=}, and \java{<}) has two corresponding bytecode opcodes, one that has two operands and one that compares a single operand to the zero value.
Table~\ref{tab:ror:opcode} shows the mappings of Java's source-level relational operators to their JVM opcodes.
For example, each instance of the \java{!=} operator in Java source code is compiled to either \jvm{IF_ICMPNE} or \jvm{IFNE} which tests for inequality between the top two values on the stack operands and inequality between the top of the stack and zero, respectively.
Our ROR mutator uses all of the bytecode-level ordered inequality operators in its implementation of \java{MethodVisitor}.

Other implementations of ROR mutators manually list every replacement opcode for each source-level relational operator~\cite{ProdigyXable}.
This results in a large amount of code duplication, much of which is unavoidable due to the \java{Map} returned being \java{static}.
However, we were able to reduce the number of changes required between each implementation by storing the bytecode relational operators in enums and iterating over the values of those enums, adding the current opcode to the \java{Map} only when the current iteration's opcode is not the opcode being replaced.
Listing~\ref{lst:ror:enum} shows an excerpt of the enumeration of the zero-comparing opcodes.
Listing~\ref{lst:ror:MethodVisitor} shows one of the enum iterations from the ROR mutator's \java{MethodVisitor}.

\begin{lstlisting}[%
  language=Java,
  frame=tb,
  caption={%
    enum excerpt from the ROR mutator.
    The \java{Opcodes} enum is imported from the package \java{org.objectweb.asm.Opcodes}.
  },
  captionpos=b,
  label={lst:ror:enum}]
enum OpcodeCompareToZero {
  IFEQ(Opcodes.IFEQ) {
    public String toString() {
      return "==";
    }
  },
  // ...
  IFNE(Opcodes.IFNE) {
    public String toString() {
      return "!=";
    }
  };

  private final int opcode;

  OpcodeCompareToZero(int opcode) {
    this.opcode = opcode;
  }

  public int getOpcode() {
    return this.opcode;
  }
}
\end{lstlisting}

\begin{lstlisting}[%
  language=Java,
  frame=tb,
  caption={%
    Excerpt of the implementation of \java{MethodVisitor} for the ROR mutator.
    The value of \java{REPLACEMENT_OP} has been previously declared to be an opcode.
  },
  captionpos=b,
  label={lst:ror:MethodVisitor}]
// The operands will seem to be in the wrong
// order when used in else conditions.
// To the bytecode parser, though, this is
// not the case.
for (OpcodeCompareToZero original
    : OpcodeCompareToZero.values()) {
  if (REPLACEMENT_OP != original) {
    MUTATIONS.put(
      original.getOpcode(),
      new Substitution(
        REPLACEMENT_OP.getOpcode(),
        "Relational operator replacement: "
        + "Mutated " + original
        + " to " + REPLACEMENT_OP));
  }
}
\end{lstlisting}

\subsection{Arithmetic Operator Replacement}
The mutation operator Arithmetic Operator Replacement (AOR) replaces basic arithmetic operators with other arithmetic operators. Pitest mutation testing tool implements only a subset of AOR, which is available within Math Mutator. The replacements available in Math Mutator are given below.\newline

\begin{tabular}{|c|c|}
	\hline
	Original Operator & Mutated Operator\\
	\hline
    + & -\\
	\hline
    - & +\\
	\hline
    * & /\\
	\hline
    / & *\\
	\hline
    \% & *\\
	\hline
\end{tabular}\\


In Phase One of this project, all missing AOR mutators are implemented. The entire replacements available in augmented PIT are summarized in below table.\\

\begin{tabular}{|c|c|}
	\hline
	Original Operator & Mutated Operators\\
	\hline
    + & -, *, /, \% \\
	\hline
    - & +, *, /, \%\\
	\hline
    * & +, -, /, \%\\
	\hline
    / & +, -, *, \%\\
	\hline
    \% & +, -, *, /\\
	\hline
\end{tabular}\\

We have implemented integer, double, float and long versions of above arithmetic mutators.



The augmented Pitest is run on 5 real world projects from GitHub and is evaluated for its functionality.
We could see increased mutation coverage and line coverage in Project Summary report available in target/pit-reports directory of the project.
We also analyzed the additional mutations generated by AOD, AOR, and ROR in different java source files and the status of the mutations, that is, Survived, Killed or No Coverage.

\section{Notable Problems}
\subsection{AOD With More Than Two Operands}
In an expression such as \java{a + b + c}, it is possible to apply AOD mutation to retain \java{a + b} and \java{c}; \java{b + c} and \java{a}; and \java{a + c} and \java{b}.
However, the current AOD implementation recursively returns \java{a}, \java{b}, \java{c} operands separately.
Adding handling of consecutive arithmetic operations may be considered for future improvements, but at the moment we do not see the significance of these mutations;
they will add more mutations to evaluate the tests but might not give better evaluation.

\subsection{Grouping The Implemented Mutations}
To register the mutations, we have to add them to Mutator.java. For AOD and AOR, this is not a problem. However for ROR, we need to test them invididually and split the enum of ROR into several parts. It is possible to group all the ENUM into one class so we do not have to add many mutations to the pom.xml files of test projects.

%%% Local Variables:
%%% mode: latex
%%% TeX-master: "../report"
%%% End:
